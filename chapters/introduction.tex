\chapter{Introduction}
\label{ch:introduction}
Our everyday lives are surrounded by plenty of devices embedding central processing units (CPU) and information storage capacities (memory).
These devices play an important role in our daily activities, such as smart phones \cite{sarwar2013impact}, as we use them to assist us on many tasks (public transport schedule information, weather, e-mail, instant messaging and so forth).
Moreover, other type of devices are taking part of these new digital world, as they are introduced in the form of \textit{"connected objects"}, which are intended to communicate not only to people through human-machine interfaces, but also to directly send and receive information from other similar objects.
Indeed, the communications means have evolved since the introduction of internet access to devices other than computers, using the same infrastructure and protocols.
This access to the Internet brought new possibilities to the devices able to implement the standard Internet Protocol (IP).
However, these new connected objects are often not able to implement this standard way of communication, due to its poor CPU power and memory constraints, relying the communication to more sophisticated devices ( \textit{i.e.} smart phones, tablets or personal computers) aiming to reach the Internet.

The \textbf{\textit{Internet of Things}} (IoT) aims to bring Internet connectivity to these resource constrained devices, by introducing more efficient protocols while being interoperable with the current IP.
Indeed, this new communication capacities make this objects reachable from anywhere, now enabled to provide services in a web fashion.
Therefore, objects being part of the IoT are intended to offer web services in a standard way, by leveraging APIs such as REST \cite{Fielding02REST}.
However, very few services on the Internet are meant to be static.
On the contrary, they are highly dynamic and tend to evolve very quickly depending on the user's needs.
This brings a big challenge for the use of constrained devices in this environment, since they are supported by operating systems and applications that were not conceived to provide dynamic characteristics.
Thus, new research challenges on how these objects can adapt to new requirements appear.
In fact, dynamic adaptation on software systems is a well known research area, and several works propose different approaches to provide solutions for this problem.
However, most of these solutions are intended for non-constrained machines such as servers and cloud computing platforms.

The main differences can be recognized as follows:
\begin{itemize}
	\item \textbf{Memory:} Kilobytes instead of Gigabytes
	\item \textbf{Energy:} Milliwatts instead of Watts
	\item \textbf{CPU:} Megahertz instead of Gigahertz 
\end{itemize}

\section{Motivations}
This research work aims to propose a way to support dynamic behaviour and self-adaptation capacities on edge IoT devices (\textit{class 2} \cite{rfc7228} as the biggest) , taking into account their very constrained resources on memory and energy autonomy.
We motivate this thesis by the real need of these devices to be highly adaptable, since their usage can vary in an important way in a short period of time.
Indeed, several domains require software that can be changed depending on various external factors, such as \textit{Smart Spaces} (Smart cities, buildings, homes, cars, and so on).
These environments are subject to human behaviour, thus requirements on the needed information from objects such as sensors, and actions performed by actuators, can vary very quickly.
Thus, the software running on these devices should be easily modifiable, without the need of physical intervention such as manual firmware change or device replacement.

The cost in time and efforts to change the software running on devices being part of a smart space can be very high, and should be reduced since estimations on the growth of IoT devices in these environments are exponential, making impossible to manually adapt each device.
Therefore, new approaches providing dynamic and automatic deployment mechanisms on IoT environments are of high interest.
Indeed, these efforts can make an IoT infrastructure a conceivable reality, bringing a more efficient use of energy for human activities, as well as a more comfortable lifestyle.

\section{Challenges}
This thesis proposes the adoption of Software Engineering approaches, and more specifically, Model Driven Engineering approaches such as Models@Runtime \cite{morin2010leveraging} to deal with the large software layer present in an IoT environment, taking into account the constrained resources and energy autonomy typical of IoT devices.

The existing approaches coming from the Software Engineering community to manage large, distributed and heterogeneous software platforms are intended for their use on powerful computers and servers, that are not aware of the memory usage and processing power needed to run their implementations.

Indeed, as the IoT can be considered as a distributed software platform, such management approaches are worth considering for its adoption and implementation on IoT devices.
However, the nature of these devices composing the IoT differs extremely from the machines on which these approaches are implemented.
IoT devices are very constrained nodes featuring few KB on RAM and some hundreds of ROM for program storage, as well as small CPUs running at very low frequencies.
Thus, direct implementations of Software Engineering approaches cannot fit these constraints.

The first challenges faced by this research work, recognized by \textit{intra-node} challenges, can be summarized on the following research questions:

\begin{itemize}
	\item RQ1: Is it possible to adapt a model@runtime approach for the software layer management on IoT environments?
	\item RQ2: Is this approach small enough on memory and CPU usage to allow scalability?
\end{itemize}

By answering these questions, we can continue to explore the possibilities given by the use of models to deal with the software layer of large distributed systems.
Indeed, the models@runtime approach proposes the use of a component model to enable adaptation features on the running platform, by modifying the reflected model and actually enact these modifications on the underlying system.
These modifications aim to affect the software component's life-cycle in order to adapt it to new requirements.

Thus, a third challenge can be highlighted for its study on this thesis:

\begin{itemize}
	\item RQ3: How can we decompose an IoT system into software components and modify its life-cycle through a model@runtime?
\end{itemize}

Finally, as decomposition of a system requires the distribution of such components among the concerned nodes, special attention should be put when they are disseminated in an IoT network.
Indeed, as the IoT network topology lacks of the robustness and bandwidth found in common Internet networks, due to the low power requirements for network interfaces, a huge amount of traffic for components distribution should be avoided.
This last challenge, the \textit{inter-node} perspective, can be represented by our last research question:

\begin{itemize}
	\item RQ4: How to \textit{efficiently} distribute, deploy and configure software components for IoT devices?
\end{itemize}

\section{Contributions}
The outcome of this thesis are two main contributions that aim to provide a complete models@runtime engine able to reconfigure and deploy software components on IoT environments.
\begin{description}
	\item[First contribution: A models@runtime engine] able to represent an IoT running application on resource constrained nodes. The transformation of the Kevoree meta-model into C code to meet the specific memory constraints of an IoT device was performed, as well as the proposition of modelling tools to manipulate a model@runtime.
	This contribution answers RQs 1 and 2.
	\item[Second contribution: Component decoupling] of an IoT system as well as an efficient component distribution algorithm. Components decoupling of an application in the context of the IoT facilitates its representation on the model@runtime, while it provides a way to easily change its behaviour by adding/removing components and changing their parameters.
	In addition, a mechanism to distribute such components using a new algorithm, called \textit{Calpulli} is proposed.
	This contribution answers RQs 3 and 4.
\end{description}

\section{Plan}
The reminder of this thesis is organized as follows:

\textbf{Chapter 2} introduces the new paradigm of the Internet of things. It describes the composition of an IoT infrastructure an gives some examples of existing similar ones.
Afterwards, we describe some current solutions aiming to support software deployment on such infrastructures.
Finally, a state of the art of the current IoT solutions and protocols is discussed.

\textbf{Chapter 3} gives the definitions of software deployment on three contexts: on monolithic, homogeneous systems, on non-monolithic heterogeneous and distributed systems, and finally on IoT systems.
Moreover, it puts special attention on the benefits of software decoupling in components, and presents an implementation example of this decomposition.
In addition, it introduces the use of models@runtime along with a component model to support Dynamic Adaptive Systems, such as the IoT.

\textbf{Chapter 4} presents our design and requirements towards an implementation of models@runtime for IoT devices.
It presents how the Kevoree approach, an implementation of the model@runtime paradigm, can be adapted for its use in constrained nodes, for which Kevoree was not initially designed.
First, we take into account the challenges of adapting such implementation in very constrained nodes, the inter-node concerns, which are not able to run JVM or code interpreters.
Second, since the networking capabilities of these devices is also limited, we need to redesign the way this models will interact with huge and constrained IoT networks, presented as the intra-node concerns.
%The minimal hardware requirements are first established to propose a first implementation, resulting in the construction of a new IoT device.
%It then describes how the constraints on memory and energy are met to support a models@runtime engine on this new device.
%, as well as an evaluation of the approach to show its feasibility and scalability on an IoT testbed.

\textbf{Chapter 5} presents the intra-node challenges and introduces the minimal requirements to run the previously designed models@runtime implementation, including the design of a representative IoT device meeting these minimal capabilities.
Afterwards, an implementation running on a typical IoT Operating System is proposed and tested on our designed device, highlighting the main challenges of the implementation.
%highlight the minimal requirements of the underlying system facilities to support the use of a model@runtime.
%A brief presentation of the challenges due to the intra-node constraints is presented, in order to provide a functional implementation.
Indeed, a large-scale method for testing is also important, thus the use of a testbed including hundreds of nodes is proposed.
Two critical needed missing features on the IoT testbed available for large scale experiments are presented, as well as the technical contributions to provide such missing features.
An evaluation of our models@runtime approach is performed to show its feasibility and scalability on a large-scale IoT testbed.

\textbf{Chapter 6} describes our final contribution, giving details about the software component decoupling of an IoT application, followed by the proposition of a new algorithm to distribute components taking into account the energy constraints and the inter-node network topology.
Two evaluations of this algorithm, an empirical and a theoretical, are performed to show its benefits.

\textbf{Chapter 7} concludes this thesis by summarizing the results of our research and highlights the benefits of using models@runtime on IoT environments to support dynamic behaviour.

\textbf{Chapter 8} gives several perspectives of future research related to this thesis.