\chapter{Models@Runtime for the IoT}
\label{ch:MARContiki}
As presented in chapter \ref{ch:IoT}, Internet of Things systems are typically formed by a myriad of many small interconnected devices.
This underlying hardware infrastructure raises new challenges in the way we administrate the software layer of these systems.
Indeed, the limited computing power and battery life of each node combined with the very distributed nature of these systems, greatly adds complexity to distributed software layer management.

In this chapter, I describe a first approach to implement a new middleware dedicated to IoT devices, in order to enable the management of software deployment and the dynamic reconfiguration of IoT systems.
Our middleware is inspired from the Component Based Systems and the model@runtime paradigm which have been already described in the previous chapter.
Moreover, the implementation of this middleware follows the directions of the existing Kevoree meta-model\todo{\cite{}}, which has been adapted to fit IoT devices constraints. 
The evaluation of these constraints were conducted on several hardware platforms typical of the IoT, in order to establish a reference for the minimum requirements to run such a middleware.
Once an initial research on these IoT platforms was conducted, the proposition of a new one was necessary, since the minimal requirements were not met by the commercially available platforms at that time.
Such platform was an essential part to evaluate our first implementation, and was used to establish the minimum requirements of an execution environment.
Finally, we have conducted an initial evaluation on an Internet of Things testbed\cite{Fleury15iotlab} recently available for experimentation.
Our results demonstrates the feasibility of providing a model@runtime middleware for these systems, which can be executed in platforms meeting the requirements already established.
This chapter is concluded by the obtained results in the FIT IoT-Lab testbed, which show the limits of our first approach.

\section{Overview}
%context
Nowadays, we are surrounded by a plethora of interconnected devices (mobile phones, household appliances, wearable sensors, and so on) that continuously collect data on our living environment.
This distributed computing infrastructure is becoming pervasive and users tend to naturally interact with it. This new computing environment offers a lot of opportunities for developing new applications in many different domains.
Technology can now operate behind the scene by dynamically responding to people wishes.
Based on this principle, IoT dedicated devices have emerged, forming the IoT infrastructure already described in section \ref{sec:IoTInfra}.
All this new IoT devices form a constellation of many small interconnected objects integrated into houses, building, cities, factory chain, etc.
%Based on this principle, Cyber Physical Systems (CPS) have emerged.CPS are pervasive and long living systems formed by a constellation of many small interconnected devices integrated into houses, building, cities, factory chain, etc. 

%motivation
In our building automation scenario presented in section \ref{sec:BAScenario}, the described IoT subsystem typically rely on sensor nodes that detect and record data such as presence, temperature, ambient lighting and energy consumption. 
Thus, the IoT uses sensors to continuously analyse the situation in order to adapt our living environment to match user needs and preferences.
User wishes may involve different objectives such as comfort, air quality, and energy savings. 
To go beyond energy management and comfort, building automation systems have to deal with new types of services, depending on the use of the building: fire safety and security management for hotel, indoor air quality control in schools and office buildings, etc. 
The opportunities of services offered by the IoT and the user preferences are countless and will change over the lifetime of these systems.
The set of small interconnected devices integrated into building can be seen as a computing infrastructure that can host these new services. 
Consequently the software deployed on these nodes needs to be dynamically reconfigured and re-deployed to meet the evolution of services and user preferences. 

% old sentences
%Consequently comfort and quality are factors of productivity and do not prevent energy savings being achieved.
%To go beyond energy management, building automation systems have to deal with new types of services, depending on the use of the building: fire safety and security management for hotel, indoor air quality control in schools and office buildings etc. 


%problem
For this reason, the capacity of dynamically deploying and reconfiguring the software layer of the IoT is a crucial feature.
We address the problem of enabling the deployment of a distributed software layer and its dynamic reconfiguration over a network of nodes featuring very limited resources: memory, processing power, bandwidth communication and energy autonomy. 

% limitation of SOTA
To solve this problem, various techniques have been proposed, such as system image replacement \cite{hui2004dynamic} and virtual machines \cite{koshy2005vmstar}. 
These approaches have two main drawbacks. 
First they are not efficient in terms of energy. 
Second, while they are suited to deploy static applications, these are not convenient solutions for IoT, since IoT devices operates in dynamic environments in which tasks performed by each node must be easily and frequently adapted.

%our approach
In this chapter we present our initial results towards the design of a middleware featuring deployment and reconfiguration facilities over an IoT system. 
Our middleware aims at implementing the paradigm of model@runtime taking into account the stringent requirements of IoT devices.

%plan
%This paper is structured as follows. Section 2 presents the Kevoree component model. Section 3 details the challenges of mapping the model@runtime paradigm to microcontrollers, and explains how we implemented Kevoree on these very limited nodes. Our proposal is evaluated in section 4. Section 5 presents related work and Section 6 gives our conclusion and highlights some perspectives to be addressed in future work.

\section{A M@R approach for constrained IoT devices}
The model@runtime paradigm has been mainly investigated in the context of distributed systems. 
These research efforts have been focused on the provision of a comprehensive set of tools to easily deploy, dynamically reconfigure, and disseminate software on a set of distributed computing units.
The current model@runtime tools have been implemented regardless of the specific characteristics and constraints of IoT devices.
In particular, the network topology and the resource constraints of the nodes forming the distributed system have not been taken into consideration.
As a result, state of the art model@runtime tools are not suitable to be used in the context of IoT Systems.
First, most approaches are relying on the Java language, which does not meet the resource constraints of the computing nodes. 
Secondly, the size of the model and its distribution among the system are not taking into consideration the limited memory capacity of each node, and their energy constraints.

In \cite{fouquet2012dynamic}, an effort has been made to port the model@runtime paradigm on the constraints of a Cyber Physical System (CPS), in which the underlying device is comparable to an IoT device.
Despite the particular attention given to the specific constraints of a Cyber Physical System, this work heavily relies on over the air firmware flashing to support the deployment and reconfiguration of software. 
We consider that relying on firmware flashing to support software deployment constitutes a flaw in the approach because of its energy cost (the complete firmware has to be sent, and if any error occurs, the whole process is restarted).
A second limitation of this approach lies in the fact that each resource constrained node relies on a more powerful node to perform most of his task related to the dynamic reconfiguration (firmware synthesis, reconfiguration decision and so on).
This second limitation is not suitable in the context of a system mainly composed of resource constrained nodes since all the resource constrained nodes have to be managed by bigger nodes.
Pushing this idea further, the management of a CPS composed of a wide number of resource constrained devices and a bigger node, the latter will have to manage all the smaller devices in a centralized management scheme.

Therefore, a new approach implementing a more complete M@R engine has been proposed.
The next section will describe an implementation based on the Kevoree meta-model, together with some tools which allow model manipulation.

\subsection{Our M@R implementation}
Since our work aims to provide a M@R engine for IoT devices, a first implementation should be developed having in mind the programming constraints.
Taking into account this limitations means that, first of all, we must fit the programming constraints.
In contrast with most of the current implementations of the M@R paradigm, which are intended for high resources machines thus they make use of high-level programming languages, our first implementation should follow a procedural language such as C.
We can justify this by the fact that most of the open source compilers for IoT devices support only this programming language, in addition to C++.
However, C++ applications are difficult to integrate in OSs like Contiki, which is one of the most convenient OS able to run our middleware.
Thus, a first approach has been developed in plain C\footnote{\url{https://github.com/kYc0o/kevoree-c}}.
This raised big challenges since the object-oriented meta-model representation is very difficult to transform into C structures, knowing that there is no modeling framework supporting such a language.

\begin{table}[htb]
	\centering
	\caption{Size of a plain C minimalistic Kevoree meta-model implmentation}
	\label{tab:kevoreeC}
	\begin{tabular}{cccccc}
		\texttt{text}   & \texttt{data} & \texttt{bss} & \texttt{dec}    & \texttt{hex}   & \texttt{filename} \\
		\texttt{181997} & \texttt{1448} & \texttt{168} & \texttt{183613} & \texttt{2cd3d} & \texttt{main}        
	\end{tabular}
\end{table}

It leads our efforts to a first manual implementation of the Kevoree meta-model.
The size of the application which contains a group and a node without components is shown in table \ref{tab:kevoreeC}






It is then necessary to characterize the minimum system requirements to implement a full model@runtime middleware, as in hardware capabilities as in execution environment.
This will avoid the use of third party nodes to perform the high level tasks described above.
However, even if a proposed device meet the characteristics to perform such tasks, the trade-offs between energy consumption and the middleware execution should be taken into account.
A general description of needed features to execute the proposed middleware is given in the next subsection.

%\subsection{Needed features of the underlying system}
\subsection{Kevoree-IoT operating system requirements}
Our middleware approach, which is called Kevoree-IoT (in reference to the kevoree-like M@R implementation of our middleware), will need several features proposed by the underlying system.
As we discussed previously, the most common approach used to run applications in IoT devices is bare-metal development followed by firmware flashing into the ROM memory.
Even if this method allows a fine control of the underlying hardware, the development time can be very long and difficult to debug, since abstractions are mostly done only at the hardware level, and does not come to the system level.
As the complexity grows, applications for IoT should be developed without special attention to hardware and system concerns.
Thus, an IoT operating system should be used, in order to leverage its system-level abstractions.
As presented in the state of the art, several IoT operating systems exist, but according to the needed features we should make a choice.
This features are the following:
\begin{enumerate}
	\item \textbf{Network stack implementing basic IP functions (TCP, UDP, HTTP, COAP).} In order to share a m@r and download software artifacts, IP communication is mandatory, since the goal is to use the Internet to reach devices from different domains.
	\item \textbf{Persistent data handling, preferably a file system.} The method used to represent a m@r is using serialized objects, usually in the JSON format. Thus, a way to store this JSON file should be provided.
	\item \textbf{Dynamic linker and loader, following the third approach presented in section \ref{sec:IoTDeployment}.} A dynamic linker and loader is essential to our approach, since changes in the model containing new software components will trigger the download and instantiation of an artifact, which should be linked and loaded at runtime by the underlying system.
	\item \textbf{Abstractions for attached devices (LEDs, sensors, actuators, etc.).} Although not mandatory, an OS usually carry basic hardware abstractions, providing an easy way to develop applications which need a physical connection with the external world.
	\item \textbf{Basic OS functionalities such as timers, task scheduler and Inter Process Communication (IPC).} Basic functionalities that are implemented very often by almost any OS.
\end{enumerate}
Given the features presented above, the system which fits all the requirements is the Contiki OS.
Despite the programming model already described as a drawback, Contiki offers all the needed functionalities, as well as a wide community which collaborate very actively in the development and debug of it.

However, even if the OS fulfill all the requirements, the computational resources needed by our middleware should be measured, in order to find a platform on which we can evaluate our approach.

\subsection{Kevoree-IoT Contiki implementation}


\subsection{Handling hardware constraints}
Designing the presented middleware in the context of the IoT raises scientific challenges.
All the following challenges are related to the limited resources of these nodes and the particular topology of the network that interconnect them.

Currently the concept of model@runtime has been applied on top of object oriented languages which offers the required features at the language level to implement the model@runtime layer. 
The first challenge lies in the fact that Contiki does not support these object oriented languages and thus the design must fit a procedural language such as C.

The second challenge lies in the very limited resources of the IoT nodes, thus forcing the middleware to meet the memory and CPU constraints. 
This imposes constraints on the way the model is represented, stored and processed on each node.
%This challenge is particularly important and hard to meet, since many software components are needed to enable the communication in these environments (radio/MAC/IP stack and so on).
Moreover the energy constraint of each IoT node, together with the mesh topology of the network makes communication in this environments fairly reliable.
This implies the necessity to optimize the way the model and the software are disseminated in the network.

At first glance, our implementation was tested on three different platforms: the \textit{\textbf{zolertia Z1\footnote{\url{http://zolertia.sourceforge.net/wiki/images/e/e8/Z1_RevC_Datasheet.pdf}}, redbee econotag\footnote{\url{http://redwire.myshopify.com/products/econotag-ii}} and wismote\footnote{\url{http://wismote.org/doku.php}}.}}
Unfortunately, none of these platforms was able to meet the space requirements in ROM and RAM to run a minimal model at runtime.

\subsection{System requirements for M@R for the IoT}
table showing the 3 available platforms z1 wismote econotag cortex-m4f

\subsubsection{First experiences with a new hardware platform}


\section{A large-scale platform for validation}

\subsection{FIT IoT-Lab}

\subsection{Extending the testbed}

\subsubsection{File system driver implementation}

\subsubsection{Runtime address relocation for ARM Cortex-M3 platforms}

\subsection{M@R experiment results in the IoT-Lab testbed}