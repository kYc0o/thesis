\chapter*{Abstract}
\markboth{Abstract}{Abstract}
\addcontentsline{toc}{chapter}{Abstract}
The Internet of Things (IoT) is covering little by little every aspect on our lives.
As this systems become more pervasive, the need of managing this complex infrastructure comes with several challenges.
Indeed, plenty of small interconnected devices are now providing more than a service in several aspects of our everyday life, which need to be adapted to new contexts without the interruption of such services.
However, this new computing system differs from classical Internet systems mainly on the type, physical size and access of the nodes.
Thus, typical methods to manage the distributed software layer on large distributed systems as usual cannot be employed on this context.
Indeed, this is due to the very different capacities on computing power and network connectivity, which are very constrained for IoT devices.
Moreover, the complexity which was before managed by experts on several fields, such as embedded systems and Wireless Sensor Networks (WSN), is now increased by the larger quantity and heterogeneity of the node's software and hardware.
Therefore, we need efficient methods to manage the software layer of these systems, taking into account the very limited resources.
%Cyber Physical Systems (CPS) or Internet of Things systems are typically formed by a myriad of many small interconnected devices.
This underlying hardware infrastructure raises new challenges in the way we administrate the software layer of these systems.
These challenges can be divided into:
\begin{itemize}
	\item \textit{intra-node}, on which we face the limited memory and CPU of IoT nodes, in order to manage the software layer
	\item and \textit{inter-node}, on which a new way to distribute the updates is needed, due to the different network topology and cost in energy for battery powered devices.
\end{itemize} 

Indeed, the limited computing power and battery life of each node combined with the very distributed nature of these systems, greatly adds complexity to the distributed software layer management.

Software reconfiguration of nodes in the Internet of Things is a major concern for various application fields.
In particular, distributing the code of updated or new software features to their final node destination in order to adapt it to new requirements, has a huge impact on energy consumption.
Most current algorithms for disseminating code over the air (OTA) are meant to disseminate a complete firmware through small chunks  and are often implemented at the network layer, thus ignoring all guiding information from the application layer.

\section*{Contribution 1: A new middleware to manage IoT Software Updates (SU)}
In this thesis we first propose a design of a new middleware dedicated to IoT devices to enable the management of software deployment and the dynamic reconfiguration of these systems, by updating its original firmware.
This middleware will act as an abstraction layer, inspired from the Models@Runtime paradigm coupled with Component Based Software Engineering (CBSE), in order to manage the inherent adaptive behaviour of an IoT system.
This new design will take into account the very constrained nature of the nodes being part of the IoT, which was not addressed by previous implementations of models@runtime and CBSE.
It includes the manual transformation into C code of a models@runtime meta-model, which represents both the network and the application layer of the Internet of Things systems, which would be part of each device's firmware.
Moreover, a new implementation is provided and an evaluation on a typical IoT infrastructure demonstrates the feasibility of providing a model@runtime middleware for these systems, addressing what we called the \textit{intra-node} challenges.
Thus distribution, instantiation and (re)configuration of the software layer, now decoupled into small pieces represented by software components, is facilitated through the abstraction layer, which can be manipulated before its actual execution on the real system.
 
\section*{Contribution 2: A new algorithm to distribute SU in IoT networks}
In our second contribution, we address the \textit{inter-node} challenges.
At runtime, we leverage the model@runtime coupled with the network information (obtained from the routing protocol) through a new algorithm to distribute software components only to those devices that need it, while adapting to a new context.
Indeed, this new algorithm, called \textit{Calpulli}, aims at minimizing energy consumption during the reconfiguration step.
We have evaluated our algorithms on representative scenario taken from real deployment scenario.
The firsts results show that our approach performs better regarding energy efficiency and deployment time in comparison to state of the art algorithms for the application layer maintenance.
Moreover, a simulation to evaluate the scalability of the approach was performed, showing that our algorithm can perform better when the network grows.